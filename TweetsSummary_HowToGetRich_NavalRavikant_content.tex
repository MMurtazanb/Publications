%%%%%%%%%%%%%%%%%%%%%%%%%%%%%%%%%%%%%%%%%%%%%%%%%%%%%%%%%%%%%%%%%%%%%%%%%%%
\begin{frame}[fragile]\frametitle{}
\begin{center}
{\Large How to Get Rich - Naval Ravikant}

{\small https://nav.al/rich}

\end{center}
\end{frame}

%%%%%%%%%%%%%%%%%%%%%%%%%%%%%%%%%%%%%%%%%%%%%%%%%%%%%%%%%%%%%%%%%%%%%%%%
\begin{frame}[fragile]
\frametitle{1. Seek Wealth, Not Money or Status}
\begin{itemize}
\item Wealth is assets that earn while you sleep; software running continuously and earning commission per transaction, money invested in mutual funds growing up, house rent, etc.
\item Wealth buys your freedom.

\item Money is how we transfer wealth
\item Status is your rank in the social hierarchy
\begin{itemize}
\item Money is not going to solve all of your problems; but it's going to solve all of your {\it money} problems.
\item Those who are not able to make money, play game of virtue, called {\it status}
\end{itemize}
\end{itemize}
\end{frame}


%%%%%%%%%%%%%%%%%%%%%%%%%%%%%%%%%%%%%%%%%%%%%%%%%%%%%%%%%%%%%%%%%%%%%%%%
\begin{frame}[fragile]
\frametitle{1. Seek Wealth, Not Money or Status}
\begin{itemize}
\item Money is not going to solve all of your problems; but it's going to solve all of your {\it money} problems.
\item Those who are not able to make money, play game of virtue, called {\it status}
\item Wealth is not a zero-sum game. Everybody in the world can have a house. My having house does not mean that you can not. But not in case of stock purchase. If some has made money in stock, someone else has lost it.
\item Status, on the other hand, is a zero-sum game. When someone is number one then the other has to be number two or lower. Both can not be number one. To be the winner, there must be a loser. That's status.
\item People creating wealth will always be attacked by people playing status games
\end{itemize}


\end{frame}

%%%%%%%%%%%%%%%%%%%%%%%%%%%%%%%%%%%%%%%%%%%%%%%%%%%%%%%%%%%%%%%%%%%%%%%%
\begin{frame}[fragile]
\frametitle{2. Make Abundance for the World}
\begin{itemize}
\item Wealth is not finite. Or else, we all would be in caves, dividing fire wood.
\item Most of the wealth in civilization has been created.
\begin{itemize}
\item Got created from somewhere. 
\item Got created from people. 
\item Got created from technology. 
\item Got created from productivity. 
\item Got created from hard work.
\end{itemize}
\item Everyone can be rich
\end{itemize}
\end{frame}

%%%%%%%%%%%%%%%%%%%%%%%%%%%%%%%%%%%%%%%%%%%%%%%%%%%%%%%%%%%%%%%%%%%%%%%%
\begin{frame}[fragile]
\frametitle{2. Make Abundance for the World}
\begin{itemize}
\item Wealth isn't about taking something from somebody else
\item Ethical wealth creation makes abundance for the world
\item Money has been historically treated as evil, because you would take/snatch money from others. Some were rich by taking work from poor. Exploitation. 
\item People got money from others by a sword, or a gun, or taxes, or crony capitalism, or Communism. Like Parasites living off a tree.
\end{itemize}
\end{frame}

%%%%%%%%%%%%%%%%%%%%%%%%%%%%%%%%%%%%%%%%%%%%%%%%%%%%%%%%%%%%%%%%%%%%%%%%
\begin{frame}[fragile]
\frametitle{2. Make Abundance for the World}
\begin{itemize}
\item At least in the developed countries, everyone is basically richer than almost anyone who was alive 200 years ago
\item We all have vehicles, medicines, food, etc.
\item ``Today, I would rather be a poor person in a First World country, than be a rich person in Louis the XIV's France''
\item In 20 years from now, most of the things would be automated. Basics are taken care. A Massive abundance.
\item We would do only creative pursuits.
\end{itemize}
\end{frame}


%%%%%%%%%%%%%%%%%%%%%%%%%%%%%%%%%%%%%%%%%%%%%%%%%%%%%%%%%%%%%%%%%%%%%%%%
\begin{frame}[fragile]
\frametitle{3. Free Markets Are Intrinsic to Humans}
\begin{itemize}
\item Capitalism [meaning free markets] is intrinsic to the human species. 
\item We cooperate. It's because we can keep track of debits and credits. Who put in how much work? Who contributed how much? That's all free market capitalism is.
\item We cant do all. So we cooperate and get what we don't make/do. We all have opportunity to produce/offer to others
\item Too many takers and not enough makers will plunge a society into ruin
\item Ideal: free minds, and free markets. Small-scale exchange between humans that's voluntary,
\end{itemize}
\end{frame}

%%%%%%%%%%%%%%%%%%%%%%%%%%%%%%%%%%%%%%%%%%%%%%%%%%%%%%%%%%%%%%%%%%%%%%%%
\begin{frame}[fragile]
\frametitle{4. Making Money Isn't About Luck}
\begin{itemize}
\item Become the kind of person who makes money and not about luck.
\item Luck should not be factor of your success at all.
\item Four types of Luck:
\begin{itemize}
\item Blind Luck: Like fortune, fate, comes without any control.
\item From Hustling (``fortune favors he brave''): Through persistence, hard work, hustle. Trying lots of things so that you get something.
\item From Preparation (``chance favors prepared mind''): Be very skilled in the field, you get better sense where fortune is.
\item From Unique Character: Because of unique/rare ability you go at such places where there is treasure, which no one can find.
\end{itemize}
\item Wealth stacks up one chip at a time, not all at once

\end{itemize}
\end{frame}

%%%%%%%%%%%%%%%%%%%%%%%%%%%%%%%%%%%%%%%%%%%%%%%%%%%%%%%%%%%%%%%%%%%%%%%%
\begin{frame}[fragile]
\frametitle{5. Make Luck Your Destiny}
\begin{itemize}
\item Build your character in a way that luck becomes deterministic, opportunity finds you.
\item Build such a strong reputation/brand that everyone wants to work with you.
\item  Dig deeper than other people do, deeper than seems rational just because you're interested. ``Extreme people get extreme results.''
\end{itemize}
\end{frame}

%%%%%%%%%%%%%%%%%%%%%%%%%%%%%%%%%%%%%%%%%%%%%%%%%%%%%%%%%%%%%%%%%%%%%%%%
\begin{frame}[fragile]
\frametitle{6. You Won't Get Rich Renting Out Your Time}
\begin{itemize}
\item You can't earn non-linearly when you're renting out your time
\item Getting paid according to hours you put.
\item When you're sleeping, you're not earning.
\item If your tasks can be trained, you are replaceable by another human or robot.
\end{itemize}
\end{frame}


%%%%%%%%%%%%%%%%%%%%%%%%%%%%%%%%%%%%%%%%%%%%%%%%%%%%%%%%%%%%%%%%%%%%%%%%
\begin{frame}[fragile]
\frametitle{7. Live Below Your Means for Freedom}
\begin{itemize}
\item People living below their means have freedom
\item Ideally, you'll make your money in discrete lumps, so that you don't upgrade your lifestyle overnight and erratically.
\end{itemize}
\end{frame}


%%%%%%%%%%%%%%%%%%%%%%%%%%%%%%%%%%%%%%%%%%%%%%%%%%%%%%%%%%%%%%%%%%%%%%%%
\begin{frame}[fragile]
\frametitle{8. Give Society What It Doesn't Know How to Get}
\begin{itemize}
\item Society will pay you for creating what it wants and delivering it at scale
\item Figure out what product you can provide and then figure out how to scale it
\item Steve Jobs figured out Smart Phone.
\item Its starts with the rich, like chauffeur driven cars, then with Uber platform, everyone has a chauffeur driven car. At scale.
\item ``Entrepreneurship is essentially an act of creating something new from scratch. Predicting that society will want it, and then figuring out how to scale it, and get it to everybody in a profitable way, in a self-sustaining way.''
\end{itemize}
\end{frame}

%%%%%%%%%%%%%%%%%%%%%%%%%%%%%%%%%%%%%%%%%%%%%%%%%%%%%%%%%%%%%%%%%%%%%%%%
\begin{frame}[fragile]
\frametitle{9. The Internet Has Massively Broadened Career Possibilities}
\begin{itemize}
\item The Internet has massively broadened the possible space of careers.
\item You can find your audience for your product, or your talent, and skill no matter how far away they are.
\item The Internet allows you to scale any niche obsession.
\item When you're competing with people it's because you're copying them. Don't.  Just do your own thing. No one can compete with you on being you. It's that simple. More authentic. 
\item No one can compete you, due to your original-ness
\end{itemize}
\end{frame}


%%%%%%%%%%%%%%%%%%%%%%%%%%%%%%%%%%%%%%%%%%%%%%%%%%%%%%%%%%%%%%%%%%%%%%%%
\begin{frame}[fragile]
\frametitle{10. Play Long-Term Games With Long-Term People}
\begin{itemize}
\item All returns in life come from compound interest in long-term games
\item Play long-term games with long-term people.
\item pick an industry in which you can play long-term games with long-term people.
\item Motivation has to come intrinsically
\item Integrity is what someone does, despite what they say they do
\item So, you want to find long-term people. You want to find people who seem irrationally ethical.
\item ``Self-esteem is the reputation that you have with yourself.'' You'll always know.
\end{itemize}
\end{frame}

%%%%%%%%%%%%%%%%%%%%%%%%%%%%%%%%%%%%%%%%%%%%%%%%%%%%%%%%%%%%%%%%%%%%%%%%
\begin{frame}[fragile]
\frametitle{12. Partner With Rational Optimists}
\begin{itemize}
\item Don't partner with cynics and pessimists; their beliefs are self-fulfilling.
\item We're genetically hardwired to be pessimists. But modern society is far, far safer. There are no tigers wandering around the street.
\item To create things, you have to be a rational optimist. Rational in the sense that you have to see the world for what it really is. And yet you have to be optimistic about your own capabilities, and your capability to get things done.
\item``Either lead, follow, or get out of the way.'' Some people have 4th option: They want to tell you why the thing is not going to work.
\end{itemize}
\end{frame}



%%%%%%%%%%%%%%%%%%%%%%%%%%%%%%%%%%%%%%%%%%%%%%%%%%%%%%%%%%%%%%%%%%%%%%%%
\begin{frame}[fragile]
\frametitle{13. Arm Yourself With Specific Knowledge}
To make money \ldots
\begin{itemize}
\item You basically get rewarded by society for giving it what it wants and it doesn't know how to get elsewhere.
\item You need:
\begin{itemize}
\item Specific Knowledge: that only you know or only a small set of people know.
\item Accountability/Brand: you want to be known as the person who can deliver that.
\item Leverage/Amplification/Multiplier-effect: You need to have ways to scale your work without you and get paid passively.
\end{itemize}
\end{itemize}
\end{frame}

%%%%%%%%%%%%%%%%%%%%%%%%%%%%%%%%%%%%%%%%%%%%%%%%%%%%%%%%%%%%%%%%%%%%%%%%
\begin{frame}[fragile]
\frametitle{Why: Specific Knowledge}
\begin{itemize}
\item If you want to make money you have to get paid at scale
\item You can stand out of the crowd and get paid sustain-ably iff you have Specific Knowledge.
\item  Much easier to be top 5 percentile at three or four things than it is to be literally the number one at something
\item Speci c knowledge comes on the job, not in a classroom
\end{itemize}
\end{frame}

%%%%%%%%%%%%%%%%%%%%%%%%%%%%%%%%%%%%%%%%%%%%%%%%%%%%%%%%%%%%%%%%%%%%%%%%
\begin{frame}[fragile]
\frametitle{What: Specific Knowledge}
\begin{itemize}
\item Is not trainable (else others can get trained and then you get reduced to margin play. It no more remains ``mine'. ONLY I CAN DO)
\item Must Scale
\item Is Passive income (Earns while you are sleeping. if you are getting paid for renting your time then thats not Specific Knowledge as it cannot scale)
\item Has Leverage (ie amplification ie once created earns multiple times based on scale)
\end{itemize}
\end{frame}

%%%%%%%%%%%%%%%%%%%%%%%%%%%%%%%%%%%%%%%%%%%%%%%%%%%%%%%%%%%%%%%%%%%%%%%%
\begin{frame}[fragile]
\frametitle{How: Specific Knowledge}
\begin{itemize}
\item Genuine curiosity
\item Building judgment in a specific domain. Have intuition built. Been there done that type confidence.
\item Experiment in your field. Gain expertize by doing. The failures are learnings too, for which you can get paid.
\end{itemize}
\end{frame}

%%%%%%%%%%%%%%%%%%%%%%%%%%%%%%%%%%%%%%%%%%%%%%%%%%%%%%%%%%%%%%%%%%%%%%%%
\begin{frame}[fragile]
\frametitle{What: Leverage}
Types:
\begin{itemize}
\item Labor: someone else working for you and expanding your work. (Difficult)
\item Capital: Expand by putting more money, systems.
\item Products that have no marginal cost of replication: books, media, movies, and code.
\end{itemize}
\end{frame}


%%%%%%%%%%%%%%%%%%%%%%%%%%%%%%%%%%%%%%%%%%%%%%%%%%%%%%%%%%%%%%%%%%%%%%%%
\begin{frame}[fragile]
\frametitle{14. Specific Knowledge Is Highly Creative or Technical}
\begin{itemize}
\item Specific knowledge is on the bleeding edge of technology, art and communication
\item Specific knowledge can be taught through apprenticeships, to an extent, on-job.
\item Specific knowledge is specific to the individual and situation
\item You can't be too deliberate about assembling specific knowledge
\item Build specific knowledge where you are a natural.
\end{itemize}
\end{frame}


%%%%%%%%%%%%%%%%%%%%%%%%%%%%%%%%%%%%%%%%%%%%%%%%%%%%%%%%%%%%%%%%%%%%%%%%
\begin{frame}[fragile]
\frametitle{15. Learn to Sell, Learn to Build}
\begin{itemize}
\item Builders: CTOs, Engineers, workers
\item Sellers: Sales, Marketing, Advertising.
\item Steve Jobs and Steve Wozniak with Apple. CEO, CTO combo.
\item If you can do both, you will be unstoppable. Elon Musk.
\item ``I'd rather teach an engineer marketing than a marketer engineering'' - Bill Gates.
\end{itemize}
\end{frame}

%%%%%%%%%%%%%%%%%%%%%%%%%%%%%%%%%%%%%%%%%%%%%%%%%%%%%%%%%%%%%%%%%%%%%%%%
\begin{frame}[fragile]
\frametitle{16. Read What You Love Until You Love to Read}
\begin{itemize}
\item You should be able to pick up any book in the library and read it
\item Avoid business magazines and business class, study microeconomics, game theory, psychology, persuasion, ethics, mathematics and computers.
\item ``reading is faster than listening, doing is faster than watching.''
\item Read the original scientific books in a field. Adam Smith's The Wealth of Nations. Darwin's Origin of the Species.
\item Don't fear any book. Pick up any book and have ability to absorb, retain good stuff and reject the bad.
\end{itemize}
\end{frame}

%%%%%%%%%%%%%%%%%%%%%%%%%%%%%%%%%%%%%%%%%%%%%%%%%%%%%%%%%%%%%%%%%%%%%%%%
\begin{frame}[fragile]
\frametitle{17. The Foundations Are Math and Logic}
\begin{itemize}
\item Mathematics and logic are the basis for understanding everything else
\item Foundational things are principles, they're algorithms.
\item It's better to read a great book slowly than to fly through a hundred books quickly.
\item Understanding comes through repetition, usage,logic and foundations that really makes you a smart thinker.
\item Learn persuasion and programming.
\item Need deep understanding of some complex topic.
\end{itemize}
\end{frame}

%%%%%%%%%%%%%%%%%%%%%%%%%%%%%%%%%%%%%%%%%%%%%%%%%%%%%%%%%%%%%%%%%%%%%%%%
\begin{frame}[fragile]
\frametitle{18. There's No Actual Skill Called `Business'}
\begin{itemize}
\item Avoid business schools and magazines
\item ``Case studies'' are just anecdotes, stories. They are other people's situations. Your stuff is different. You will understand them only when you go through them in real life.
\item Doing is faster than watching or listening. You cant change speed too much. You can not go back. Hard to highlight, save to notebooks, etc.
\item The number of `doing' iterations drives the learning curve. But not repetition of same things. Iterations are upgrades every time. New weights after learning from back-propagation. Changed approach every time.
\item If you're willing to bleed a little every day, you may win big later
\end{itemize}
\end{frame}


%%%%%%%%%%%%%%%%%%%%%%%%%%%%%%%%%%%%%%%%%%%%%%%%%%%%%%%%%%%%%%%%%%%%%%%%
\begin{frame}[fragile]
\frametitle{19. Embrace Accountability to Get Leverage}
\begin{itemize}
\item Take risks under your own name and society will reward you with leverage.
\item People who can fail in public have a lot of power.
\item Clear accountability is important. Without accountability, you don't have incentives, credibility.
\item A well-functioning team has clear accountability for each position.
\end{itemize}
\end{frame}


%%%%%%%%%%%%%%%%%%%%%%%%%%%%%%%%%%%%%%%%%%%%%%%%%%%%%%%%%%%%%%%%%%%%%%%%
\begin{frame}[fragile]
\frametitle{20. Take Accountability to Earn Equity}
\begin{itemize}
\item If you have high accountability, you're less replaceable
\item Accountability is how you're going to get equity.
\item The equity holders take on greater risk, but in exchange, they get nearly unlimited upside.
\item Accountability is reputational skin in the game
\end{itemize}
\end{frame}

%%%%%%%%%%%%%%%%%%%%%%%%%%%%%%%%%%%%%%%%%%%%%%%%%%%%%%%%%%%%%%%%%%%%%%%%
\begin{frame}[fragile]
\frametitle{21. Labor and Capital Are Old Leverage}
\begin{itemize}
\item Everyone is fighting over labor and capital
\item Our brains aren't evolved to comprehend new forms of leverage
\item ``Fortunes require leverage. Business leverage comes from capital, people and products with no marginal costs of replication.''
\item Society overvalues labor leverage. To get more done, I employ more people.
\item Need actually less labor and more of other types of leverages to excel.
\item Managing people is hard. Avoid as much as possible.

\end{itemize}
\end{frame}

%%%%%%%%%%%%%%%%%%%%%%%%%%%%%%%%%%%%%%%%%%%%%%%%%%%%%%%%%%%%%%%%%%%%%%%%
\begin{frame}[fragile]
\frametitle{21. Labor and Capital Are Old Leverage}
\begin{itemize}
\item Capital has been the dominant form of leverage in the last century
\item With money you can buy better technology (which is generally expensive).
\item If invested well, can earn great amplification. And then there is compounding effect.
\item You need specific knowledge and accountability to obtain capital
\item You get money from other based on your reputation.

\end{itemize}
\end{frame}

%%%%%%%%%%%%%%%%%%%%%%%%%%%%%%%%%%%%%%%%%%%%%%%%%%%%%%%%%%%%%%%%%%%%%%%%
\begin{frame}[fragile]
\frametitle{22. Product and Media Are New Leverage}
\begin{itemize}
\item Create software and media that work for you while you sleep
\item They have no marginal cost of replication. Meaning, you can multiply your efforts without having to involve other humans and without needing money from other humans.
\item All new fortunes can be made here.
\item Combining all three forms of leverage is a magic combination, a startup, less but very smart engineers, VC capital for marketing/scaling and lots of code/media/content.
\item No permissions are needed use Code/media leverage.
\end{itemize}
\end{frame}

%%%%%%%%%%%%%%%%%%%%%%%%%%%%%%%%%%%%%%%%%%%%%%%%%%%%%%%%%%%%%%%%%%%%%%%%
\begin{frame}[fragile]
\frametitle{23. Product Leverage is Egalitarian}
\begin{itemize}
\item The best products tend to be available to everyone
\item Product leverage is a positive-sum game
\item Worlds richest are using same Android OS as others. The Flat world!!
\item Status goods are limited to a few people. But functionally need not be too great. Rolex watch still shows the same time.
\item The best products tend to be targeted at the middle class.
\item Creating wealth with product leads to more ethical wealth.
\end{itemize}
\end{frame}

%%%%%%%%%%%%%%%%%%%%%%%%%%%%%%%%%%%%%%%%%%%%%%%%%%%%%%%%%%%%%%%%%%%%%%%%
\begin{frame}[fragile]
\frametitle{24. Pick a Business Model with Leverage}
\begin{itemize}
\item An ideal business model has network effects, low marginal costs and scale economies.
\item Network effects: value grows as the square of the customers
\item Zero marginal cost of reproduction: producing more is free
\item Scale economies: the more you produce, the cheaper it gets
\item In a network effect, each new user adds value to the existing users
\end{itemize}
\end{frame}

%%%%%%%%%%%%%%%%%%%%%%%%%%%%%%%%%%%%%%%%%%%%%%%%%%%%%%%%%%%%%%%%%%%%%%%%
\begin{frame}[fragile]
\frametitle{25. Example: From Laborer to Entrepreneur}
Example: Housing
\begin{itemize}
\item Laborers get paid hourly and have low accountability
\item General contractors get equity, but they're also taking risk
\item Property developers pocket the profit by applying capital leverage
\item Architects, large developers and REITs are even higher in the stack
\item Real estate tech companies apply the maximum leverage.
\end{itemize}
\end{frame}

%%%%%%%%%%%%%%%%%%%%%%%%%%%%%%%%%%%%%%%%%%%%%%%%%%%%%%%%%%%%%%%%%%%%%%%%
\begin{frame}[fragile]
\frametitle{26. Judgment Is the Decisive Skill}
\begin{itemize}
\item If you're steering a big ship, if you're steering Google or Apple, and your judgment is 10 or 20 percent better than the next person's, society will literally pay you hundreds of millions of dollars more, because you're steering a \$100 billion ship.
\item Judgment is knowing the long-term consequences of your actions
\item Without experience, judgment is often less than useless.
\item The people with the best judgment are among the least emotional
\item Emotions are what prevent you from seeing what's actually happening
\item Judgment is the exercise of wisdom. Wisdom comes from experience; and that experience can be accelerated through short iterations.
\item Top investors often sound like philosophers
\end{itemize}
\end{frame}

%%%%%%%%%%%%%%%%%%%%%%%%%%%%%%%%%%%%%%%%%%%%%%%%%%%%%%%%%%%%%%%%%%%%%%%%
\begin{frame}[fragile]
\frametitle{27. Set an Aspirational Hourly Rate}
\begin{itemize}
\item Set and enforce an aspirational hourly rate.
\item Outsource tasks that cost less than your hourly rate.
\item ``My aspirational rate was \$5,000/hr'' - Naval
\item Btw, Naval, after extrapolation to year (which is multi million dollars), achieved this!!
\item ``You should be working on your product and
getting product-market fit, and you should be exercising and eating healthy. That's about it. That's all you have time for while you're on this mission.''- Paul Graham
\end{itemize}
\end{frame}

%%%%%%%%%%%%%%%%%%%%%%%%%%%%%%%%%%%%%%%%%%%%%%%%%%%%%%%%%%%%%%%%%%%%%%%%
\begin{frame}[fragile]
\frametitle{28. Work As Hard As You Can}
\begin{itemize}
\item If getting wealthy is your goal, you're going to have to work as hard as you can. But hard work is no
substitute for who you work with and what you work on. Those are the most important things.
\item  ``product-market-founder  fit'' taking into account how well a founder is personally suited to the business.
\item No matter how high your bar is, raise it
\item Anything you have to do, get it done. Why wait? You're not getting any younger.
\item Impatience with actions, patience with results.
\end{itemize}
\end{frame}

%%%%%%%%%%%%%%%%%%%%%%%%%%%%%%%%%%%%%%%%%%%%%%%%%%%%%%%%%%%%%%%%%%%%%%%%
\begin{frame}[fragile]
\frametitle{29. Be Too Busy to ‘Do Coffee'}
\begin{itemize}
\item Ruthlessly decline meetings
\item Be too busy to `do coffee' while keeping an uncluttered calendar.
\item If someone wants a meeting, see if they will do a call instead. If they want to call, see if they will email
instead. If they want to email, see if they will text instead. And you probably should ignore most text
messages—unless they're true emergencies.
\item When you do meetings, make them walking
meetings. Do standing meetings. Keep them short, actionable and small. 
\item People will meet with you when you have proof of work, ie somthing working, not for ppt.
\item  Product progress is the entrepreneur's resume. It's an unfake-able resume.
\item Free your time and mind.
\end{itemize}
\end{frame}

%%%%%%%%%%%%%%%%%%%%%%%%%%%%%%%%%%%%%%%%%%%%%%%%%%%%%%%%%%%%%%%%%%%%%%%%
\begin{frame}[fragile]
\frametitle{30. Keep Redefining What You Do}
\begin{itemize}
\item Become the best in the world at what you do.
\item Keep redefining what you do until you're the best at what you do.
\item You want to be number one. And you want to keep changing what you do until you're number one.
\item Your objective and skills should converge to make you number one.
\item If you want to be successful in life, you have to get comfortable managing multi-variate problems
and multiple-objective functions at once. 
\end{itemize}
\end{frame}

%%%%%%%%%%%%%%%%%%%%%%%%%%%%%%%%%%%%%%%%%%%%%%%%%%%%%%%%%%%%%%%%%%%%%%%%
\begin{frame}[fragile]
\frametitle{31. Escape Competition Through Authenticity}
\begin{itemize}
\item Nobody can compete with you on being you.
\item Competition will trap you in a lesser game
\item Get away from the specter of competition, which is not just stressful and nerve-wracking but also will drive you to the wrong answer—is to be authentic to yourself.
\item If someone else starts firing rockets, Elon won't mind. He has his own mission. Mars. Whatever others may be doing, he is on it.
\end{itemize}
\end{frame}


%%%%%%%%%%%%%%%%%%%%%%%%%%%%%%%%%%%%%%%%%%%%%%%%%%%%%%%%%%%%%%%%%%%%%%%%
\begin{frame}[fragile]
\frametitle{32. Play Stupid Games, Win Stupid Prizes}
\begin{itemize}
\item Competition will blind you to greater games.
\item Businesses that seem like they're in direct competition really aren't. Me too businesses. Shallow margins. Volume - TCSs, Wipros, INfys or the world.
\item They end up adjacent or slightly
different. You're one step away from a completely different business, and sometimes you need to take
that step. You're not going to take it if you're busy  ghting over a booby prize.
\end{itemize}
\end{frame}

%%%%%%%%%%%%%%%%%%%%%%%%%%%%%%%%%%%%%%%%%%%%%%%%%%%%%%%%%%%%%%%%%%%%%%%%
\begin{frame}[fragile]
\frametitle{33. Eventually You Will Get What You Deserve}
\begin{itemize}
\item On a long enough timescale, you will get paid.
\item It takes time.
\item  Apply judgment, apply accountability, and apply the skill of reading.
\item What are you really good at, that the market values?
\end{itemize}
\end{frame}

%%%%%%%%%%%%%%%%%%%%%%%%%%%%%%%%%%%%%%%%%%%%%%%%%%%%%%%%%%%%%%%%%%%%%%%%
\begin{frame}[fragile]
\frametitle{34. Reject Most Advice}
\begin{itemize}
\item Most advice is people giving you their winning lottery ticket numbers. It wont make you rich.
\item The best founders listen to everyone but make up their own mind.
\item Scott Adams - ``systems not goals''
\item Even these notes,  you should examine everything. If something doesn't feel true to you, put it down.
Set it aside.
\item Advice offers anecdotes to recall later, when you get your own experience
\end{itemize}
\end{frame}

%%%%%%%%%%%%%%%%%%%%%%%%%%%%%%%%%%%%%%%%%%%%%%%%%%%%%%%%%%%%%%%%%%%%%%%%
\begin{frame}[fragile]
\frametitle{35. A Calm Mind, a Fit Body, a House Full of Love}
\begin{itemize}
\item When you're wealthy, you'll realize it wasn't what you were seeking.
\item Yes, money will solve all your money problems. But it doesn't get you everywhere. 
\item You're still the same person.  If
you're happy, you're happy. If you're unhappy, you're unhappy. If you're calm and fulfilled and peaceful,
you're still that same person.
\item A calm mind, a  fit body and a house full of love must be earned
\item A lot of divorces happen over money, a lot of battles happen over internal anger
\end{itemize}
\end{frame}

%%%%%%%%%%%%%%%%%%%%%%%%%%%%%%%%%%%%%%%%%%%%%%%%%%%%%%%%%%%%%%%%%%%%%%%%
\begin{frame}[fragile]
\frametitle{36. There Are No Get Rich Quick Schemes}
\begin{itemize}
\item Get rich quick schemes are just someone else getting rich off  you.
\item We don't have ads because it would ruin our credibility. same for me on LinkedIn. I don't post someone else's job requirements. That smells foul money deal.
\item . If I say, ''I know how to get rich, and I'm going to sell that to you,'' then it ruins it.
\item Anyone giving advice on how to get rich should have made their money elsewhere
\end{itemize}
\end{frame}

%%%%%%%%%%%%%%%%%%%%%%%%%%%%%%%%%%%%%%%%%%%%%%%%%%%%%%%%%%%%%%%%%%%%%%%%
\begin{frame}[fragile]
\frametitle{37. Productize Yourself}
\begin{itemize}
\item Figure out what you're uniquely good at, and apply as much leverage as possible
\item Find hobbies that make you rich, fit and creative.
\item One that makes you money, one that makes you  fit, and one that
makes you smarter. 
\item say, Yoga can keep you fit throughout your life.
\end{itemize}
\end{frame}

%%%%%%%%%%%%%%%%%%%%%%%%%%%%%%%%%%%%%%%%%%%%%%%%%%%%%%%%%%%%%%%%%%%%%%%%
\begin{frame}[fragile]
\frametitle{38. Accountability Means Letting People Criticize You}
\begin{itemize}
\item You have to stick your neck out and be willing to fail publicly
\item Accountability means letting people criticize you.
\item Don't refuse to do things just because others can't do them
\item Realize your philanthropic vision by running a business.
\item Many non-profit efforts would be better off  as for-profit companies.
\end{itemize}
\end{frame}

%%%%%%%%%%%%%%%%%%%%%%%%%%%%%%%%%%%%%%%%%%%%%%%%%%%%%%%%%%%%%%%%%%%%%%%%
\begin{frame}[fragile]
\frametitle{39. We Should Eventually Be Working for Ourselves}
\begin{itemize}
\item Midlife can be the most fruitful time to apply this advice
\item But we will have to make sacrifices and take on more risk.
\item Look up the value chain to find leverage
\item You will do better in a small organization
\item The goal is that we are all working for ourselves
\end{itemize}
\end{frame}

%%%%%%%%%%%%%%%%%%%%%%%%%%%%%%%%%%%%%%%%%%%%%%%%%%%%%%%%%%%%%%%%%%%%%%%%
\begin{frame}[fragile]
\frametitle{40. Being Ethical Is Long-Term Greedy}
\begin{itemize}
\item If you cut fair deals, you will get paid in the long run
\item Ethics isn't something you study; it's something you do
\item Trust leads to compounding relationships
\item Being ethical attracts other long-term players
\item Being ethical is long-term greedy
\end{itemize}
\end{frame}

%%%%%%%%%%%%%%%%%%%%%%%%%%%%%%%%%%%%%%%%%%%%%%%%%%%%%%%%%%%%%%%%%%%%%%%%
\begin{frame}[fragile]
\frametitle{41. Envy Can Be Useful, or It Can Eat You Alive}
\begin{itemize}
\item Suffering through the wrong thing can motivate you to  find the right thing
\item But there are points in your life when it can be a powerful booster rocket.
\end{itemize}
\end{frame}

%%%%%%%%%%%%%%%%%%%%%%%%%%%%%%%%%%%%%%%%%%%%%%%%%%%%%%%%%%%%%%%%%%%%%%%%
\begin{frame}[fragile]
\frametitle{42. Principal-Agent Problem: Act Like an Owner}
\begin{itemize}
\item If you think and act like an owner, it's only a matter of time until you become an owner
\item A principal is an owner; an agent is an employee
\item A principal's incentives are different than an agent's incentives.
\item The principal/owner of the business wants what is best for the business and will make the most money. 
\item  The agent generally wants whatever will look good to the principal, or might make them the most friends in the neighborhood or in the business, or might make them personally the most money.
\end{itemize}
\end{frame}

%%%%%%%%%%%%%%%%%%%%%%%%%%%%%%%%%%%%%%%%%%%%%%%%%%%%%%%%%%%%%%%%%%%%%%%%
\begin{frame}[fragile]
\frametitle{42. Principal-Agent Problem: Act Like an Owner}
\begin{itemize}
\item Almost all human behavior can be explained by incentives.
\item When you do deals, it’s better to have the same incentives
\item  You want to be generous with your top lieutenants—in terms of ownership and incentives—even if they don’t necessarily
realize it; because over time they will and you want them to be aligned with you.
\item If you’re an employee, your most important job is to think like a principal
\item In big girms, Principal and Agent are very separated. Deal with small  rms to avoid the principal-agent problem
\end{itemize}
\end{frame}


%%%%%%%%%%%%%%%%%%%%%%%%%%%%%%%%%%%%%%%%%%%%%%%%%%%%%%%%%%%%%%%%%%%%%%%%
\begin{frame}[fragile]
\frametitle{43. Kelly Criterion: Avoid Ruin}
\begin{itemize}
\item  Don’t risk everything. Stay out of jail. Don’t bet everything on one big gamble. Be careful how
much you bet each time, so you don’t lose the whole kitty.
\item Nassim Taleb: `` ergodicity'': What is true
for 100 people on average isn’t the same as one person averaging that same thing 100 times.
\item So risk-taking—especially when the averages that are
calculated across large populations—is not always rational. If chances of winning are 50\% over population, then on a day, you may just lose, fully, thats it. Cant play anymore.
\item Dont take risks,  dont cut corners and doing unethical or downright illegal things. You will be in jail for a small risk/gain.
\item Ruining your reputation is the same as getting wiped to zero
\end{itemize}
\end{frame}

%%%%%%%%%%%%%%%%%%%%%%%%%%%%%%%%%%%%%%%%%%%%%%%%%%%%%%%%%%%%%%%%%%%%%%%%
\begin{frame}[fragile]
\frametitle{44. Schelling Point: Cooperating Without Communicating}
\begin{itemize}
\item In a multiplayer game, where people respond based on what they think the other person’s
response will be.
\item Shelling formulation:  How do you get people who
cannot communicate with each other to coordinate?
\item Competing companies can settle at common price without communication, assuming both are rational.
\item Use social norms to cooperate when you can’t communicate
\end{itemize}
\end{frame}

%%%%%%%%%%%%%%%%%%%%%%%%%%%%%%%%%%%%%%%%%%%%%%%%%%%%%%%%%%%%%%%%%%%%%%%%
\begin{frame}[fragile]
\frametitle{45. Turn Short-Term Games Into Long-Term Games}
\begin{itemize}
\item Improve your leverage by turning short-term relationships into long-term ones
\item Pareto superior means something is better in some ways while being equal or better in other ways. It’s
not worse in any way. Useful in negotiations.
\item Pareto optimal is when the solution is the best it can possibly be and you can’t change it without making
it worse in at least one dimension. There is a hard trade-o  from this point forward.
\item . If you want something too badly, the other person can extract more value from you.
\item if someone is taking advantage, make short-term to long term. Make it repeat game.
\item  Try to bring reputation into the negotiation.
Try to include other people who may want to play games with this person in the future.
\end{itemize}
\end{frame}

%%%%%%%%%%%%%%%%%%%%%%%%%%%%%%%%%%%%%%%%%%%%%%%%%%%%%%%%%%%%%%%%%%%%%%%%
\begin{frame}[fragile]
\frametitle{45. Turn Short-Term Games Into Long-Term Games}
\begin{itemize}
\item Convert single-move games to multi-move games
\item  “Actually, I need two different projects done. The  first project we’ll do
together, and based on that I’ll decide if we do the second project.”
\item  “I’m going to do this project with you, and I have three friends who want projects
done who are waiting to see the outcome of this project.”
\item Another way to address, you being taken advantage of, is to write public review.
\end{itemize}
\end{frame}


%%%%%%%%%%%%%%%%%%%%%%%%%%%%%%%%%%%%%%%%%%%%%%%%%%%%%%%%%%%%%%%%%%%%%%%%
\begin{frame}[fragile]
\frametitle{46. Compounding Relationships Make Life Easier}
\begin{itemize}
\item Life gets a lot easier when you know someone’s got your back
\item Mutual trust makes it easy to do business
\item If you are dealing with someone of 20 years of trust, you dont read contracts.
\item The most under-recognized reason startups fail is because the founders fall apart.
\item It’s better to have a few compounding relationships than many shallow ones
\end{itemize}
\end{frame}

%%%%%%%%%%%%%%%%%%%%%%%%%%%%%%%%%%%%%%%%%%%%%%%%%%%%%%%%%%%%%%%%%%%%%%%%
\begin{frame}[fragile]
\frametitle{47. Price Discrimination: Charge Some People More}
\begin{itemize}
\item You can charge people for extras based on their propensity/ability to pay
\item Price discrimination is a technique for charging certain people more, with something extra.
\item Business-class seats routinely cost  ve or 10 times more than economy seats. But it costs the airline
much less—maybe two or three times more than a standard seat—to provide perks like wider seats, more
legroom and free drinks.
\item Rich people and large enterprises are willing to pay more, so price discrimination works. They need a bit extra facility and status signal.
\end{itemize}
\end{frame}

%%%%%%%%%%%%%%%%%%%%%%%%%%%%%%%%%%%%%%%%%%%%%%%%%%%%%%%%%%%%%%%%%%%%%%%%
\begin{frame}[fragile]
\frametitle{48. Consumer Surplus: Getting More Than You Paid For}
\begin{itemize}
\item People are willing to pay more than what companies charge
\item Consumer surplus is the excess
value you get from something when you pay less than you were willing to pay.
\item Although well-to-do people can pay \$20 for Starbucks coffee, it is still kept at \$5 for general consumers. So rich are  getting a lot of consumer surplus out of the coffee.  
\item A lot of
people are willing to pay more than what Amazon charges.
\end{itemize}
\end{frame}

%%%%%%%%%%%%%%%%%%%%%%%%%%%%%%%%%%%%%%%%%%%%%%%%%%%%%%%%%%%%%%%%%%%%%%%%
\begin{frame}[fragile]
\frametitle{49. Net Present Value: What Future Income Is Worth Today}
\begin{itemize}
\item See what future income is worth today by applying a discount to its future value
\item NPV: That stream of payments I’m going to get in the future—
what’s it worth today?
\item Apply this for any future payments, valuations, corpus needed to survive, etc.
\end{itemize}
\end{frame}

%%%%%%%%%%%%%%%%%%%%%%%%%%%%%%%%%%%%%%%%%%%%%%%%%%%%%%%%%%%%%%%%%%%%%%%%
\begin{frame}[fragile]
\frametitle{50. Externalities: Calculating the Hidden Costs of Products}
\begin{itemize}
\item Externalities let you account for the true cost of products by including hidden costs
\item  An externality is where there’s an additional cost imposed by whatever product is being produced
or consumed, that’s not accounted for in the price of the product. Say, destroying environment.
\item Pricing externalities properly is more effective than feel-good measures. Because the environment is  finite and precious, we have to price it properly and fold that back into the
cost of products and services.
\end{itemize}
\end{frame}


%%%%%%%%%%%%%%%%%%%%%%%%%%%%%%%%%%%%%%%%%%%%%%%%%%%%%%%%%%%%%%%%%%%%%%%%
\begin{frame}[fragile]
\frametitle{Bonus: Finding Time to Invest in Yourself}
\begin{itemize}
\item If you have to work a “normal job,” take on accountability to build your speci c knowledge
\item You will need to rent your time to get started.
This is only acceptable when you are learning and saving.
\item Chief of staff  for a founder is one of the most coveted jobs for young people starting out in Silicon Valley.
\item Find the part of the job with the steepest learning curve
\item Develop a founder mentality
\item Accountability is something you can take on immediately
\end{itemize}
\end{frame}

%%%%%%%%%%%%%%%%%%%%%%%%%%%%%%%%%%%%%%%%%%%%%%%%%%%%%%%%%%%%%%%%%%%%%%%%
\begin{frame}[fragile]
\frametitle{Bonus: Finding Time to Invest in Yourself}
\begin{itemize}
\item Specific knowledge can be timely or timeless
\item If you become a world-class expert in machine learning just as it takes o  and you got there through
genuine intellectual interest, you’re going to do really well. But 20 years from now, machine learning may
be second hat; the world may have moved on to something else. That’s timely knowledge.
\item If you’re good at persuading people, it’s probably a skill you picked up early on in life. It’s always going to
apply, because persuading people is always going to be valuable. That’s timeless knowledge.
\item You need combo. Scott Adams writes. You might combine persuasion with accounting and an
understanding of semiconductor production lines. Now you can become the best semiconductor
salesperson and, later on, the best semiconductor company CEO.

\end{itemize}
\end{frame}

%%%%%%%%%%%%%%%%%%%%%%%%%%%%%%%%%%%%%%%%%%%%%%%%%%%%%%%%%%%%%%%%%%%%%%%%
\begin{frame}[fragile]
\frametitle{Bonus: Finding Time to Invest in Yourself}
\begin{itemize}
\item Danny Hillis famously said technology is everything that doesn’t work yet.
\item Technology is, by de nition, the intellectual frontier. It’s taking things from science and culture that we
have not  gured out how to mass produce or create e ciently and  guring out how to commercialize it
and make it available to everybody.
\item Technology will always be a great  eld where you can pick up speci c knowledge that is valuable to
society.
\end{itemize}
\end{frame}

%%%%%%%%%%%%%%%%%%%%%%%%%%%%%%%%%%%%%%%%%%%%%%%%%%%%%%%%%%%%%%%%%%%%%%%%
\begin{frame}[fragile]
\frametitle{Bonus: Finding Time to Invest in Yourself}
\begin{itemize}
\item If you don’t have accountability, do something different
\item Companies don’t know
how to measure outputs, so they measure inputs instead. Work in a way that your outputs are visible and
measurable. 
\item   From the agricultural and industrial ages: The amount of hours you put into doing something was a reliable
proxy for what kind of output you’d get.  its linear.
\item Today, it’s extremely nonlinear. One good investment decision can make a company $10 million or $100
million. One good product feature can be the difference between product-market fit and complete failure
\end{itemize}
\end{frame}

%%%%%%%%%%%%%%%%%%%%%%%%%%%%%%%%%%%%%%%%%%%%%%%%%%%%%%%%%%%%%%%%%%%%%%%%
\begin{frame}[fragile]
\frametitle{Bonus: Finding Time to Invest in Yourself}
\begin{itemize}
\item As a result, judgment and accountability matter much more. Often the best engineers aren’t the hardest
workers. Sometimes they don’t work very hard at all, but they dependably ship that one critical product
at just the right time
\item People need to be able to tell what role you had in the company’s success. 
\item You’ll develop thick-skin if you take on accountability, as both good and bad hings come yuor way.
\end{itemize}
\end{frame}
