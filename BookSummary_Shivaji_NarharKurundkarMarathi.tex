\begin{center}
     \Large{\textbf{छत्रपती शिवाजी महाराज जीवन रहस्य \\ नरहर कुरुंदकर}}  % Change Title
\end{center}

{\em प्रख्यात विचारवंत प्रा. कुरुंदकरांनी परंपरेनुसार शिवाजीच्या जीवनातील लढाया, सुटका इत्यादी प्रसिद्ध-रोमांचकारी प्रसंगात न जाता त्याच्या कर्तुत्वामागील खरी प्रेरणा कोणती व त्याच्याबद्दल लोकांच्या मनातील देवत्वाची कारणे शोधण्याचा प्रयत्न केला आहे}

%\section{टिपणे}
\bigskip

\begin{itemize}[noitemsep,nolistsep]

\item पूर्वी अनेक राजे प्रजेचे हितकारी, पुण्यश्लोक म्हणून गणले गेले, पण समकालीनांनी व उत्तरकालीनांनी ज्याला ईश्वरी अवतार मानला तो एकटा शिवाजीच!!

\item महाराज्यांच्या जीवनात रोमांचकारी घटनांचा कालखंड फार छोटा (अफझलखान वध ते आग्य्राहून सुटका, साधारण ७ वर्षे) आहे. या घटनांच्या अलीकडची २९ व नंतरची १४ वर्षे नाट्यमय-शून्य आहेत. या अ-रोमांचकारी काळातील शिवाजीचे काम त्याचे कर्तुत्व-देवत्व समजण्यास उपयोगी पडते. राष्ट्राची उभारणी चिवटपणे, सातत्याने, दीर्घकाळ सामान्य घटनांमधून होत असते याचे हे उदाहरणच आहे.

\item शिवाजीचा लढा हिंदू विरुद्ध मुसलमान असा नव्हता. शिवाजीच्या सैन्यात मुसलमान अधिकारी व सैनिक होते. कुतुबशहाशी त्याचे संबंध मित्रत्वाचे होते. तर त्याला ज्यांच्या विरुद्ध प्रामुख्याने लढावे लागले ते त्याच्या धर्मातीलच नाही तर जातीतील- नातेवाईक होते. 

\item हे राज्य जनतेसाठी व्हावे ही भूमिका होती. त्यानी हे स्वत:चे न मानता परमेश्वराचे मानले आहे. जनेतेनेही हे राज्य आपले मानले. माझे राज्य आहे, कोणी नेता असो व नसो मला हे प्राणपणाने टिकवले पाहिजे असे जनतेला वाटले.

\item १६४२-४७ : दादोजींच्या नेतृत्वाखाली १२ मावळ ताब्यात घेणे व तिथे व्यवस्था लावणे हा उपक्रम. तेंव्हाचे वतनदार मन मानेल तशी वसुली करत, जनतेला लुबडीत, स्वत: भोग विलासात राहत, त्यांचा बंदोबस्त करून शिवाजीने जनतेचा दुवा घेतला.

\item १६४५-४९: उठावाचा काळ. 
	\begin{itemize}[noitemsep,nolistsep]
	\item तोरणा-रोहीडा असे किल्ले जिंकून नव्या ताब्यात घेतलेल्या प्रदेशाची व्यवस्था लावणे. जुलूम करणाऱ्यांना कठोर शासन. प्रत्येक अत्याचारी पुरुषाच्या अवयवाचा शिवाजीने पडलेला तुकडा त्याला दहा निष्ठावंत मिळवून देत असेल. शिवाजी म्हणजे नव्या व्यवस्थेचा आग्रह!! 
	\item छोट्या लढाया-विजय यांनी आत्मविश्वास. मोठे युद्ध टाळण्यासाठी सिंहगड रिकामा करून दिला व विजापूरकारांशी सलोखा. वैशिष्टय असे की, शिवाजीला जरी काही वेळेस प्रभाव पत्करावा लागला, माघार घ्यावी लागली, प्रदेश होरपळून निघाला तरी लोकांचा विश्वास उडाला नाही, कोणीही (एखादा चुकार सोडल्यास) बंड केले नाही. शिवाजीच्या इमानावर लोक मन ठेवून आले.
	\end{itemize}

\item १६४९-५६: शांततेचा, निर्दोष प्रदेश व्यवस्थेच्या प्रयोगांचा काळ. जनतेचे बळ व त्या ओघाने फौजेचे बळ वाढवले
\item १६५६-५९  : आदिलशाहीला आव्हान
	\begin{itemize}[noitemsep,nolistsep]
	\item वेगाने जावळी ताब्यात घेतली. सिंहगड घेऊन आदिलशहाला जाहीर आव्हान. प्रदेश वाढवत समुद्रापर्यंत सीमा भिडवल्या. उद्या गरज पडली तर जावळीच्या आसपासचा प्रदेश सोडून देऊन तह करण्याची पूर्वतयारी पण केली. दारुण पराभवाच्या उंबरठ्यावर शिवाजी प्रदेश सोडतो आणि लढण्याची क्षमता सुरक्षित ठेवतो.
	\item अफजलखानाचा वध: प्रथम दगा कोणी दिला हा प्रश्न निरर्थक आहे, प्रतापगडाशी आले खान हा जिवंत जाणारच नव्हता. दग्याने ठार करणे हा खानचा इतिहास होता. सर्व बाजूनी फौजा घुसवून, बेचिराख करून तो गोंधळ माजवत होता. खानला जावळीच्या खोऱ्यात आणून मारायचे हा एका भव्य योजनेतील छोटा भाग होता. त्याची बारा हजाराची फौजही मारायची होती.
	\item खानवधानंतर महाराजांनी प्रचंड लुट करून मोठी सेना उभी केली. घोडदळ दुप्पट केले. आदिलशाही या वधाच्या धक्क्यातून सावरण्याच्या आत महाराजांनी लगेच वाई घेतली, पन्हाळगड घेतला. अखिल भारताला महाराजांचा परिचय झाला.
	\end{itemize}

\item १६५९-१६७०: मोठी आव्हाने
	\begin{itemize}[noitemsep,nolistsep]
	\item सिद्दी जोहर हे खानवधाला आदिलशाहीने दिलेले उत्तर आहे. त्याने शिवाजी पन्हाळगडावर असताना वेढा देण्यात यश मिळवले. सिद्दीने स्वत:च्या रणांगणात शिवाजीला अडकवले होते व त्यातून सोडवणे मराठयांच्या ताकदीबाहेरचे होते. याच वेळी शिवाजीचा अंदाज (की औरंगजेब आपल्याला साधा-स्थानिक जहागीरदार समजून काही करणार नाही) चुकला व शाहीस्तेखानाला चालून आला. नंतर मिर्झा राजे जयसिंग यांनी नवीन भर घातली.
	\item मरण्याची प्रेरणा:
		\begin{itemize}[noitemsep,nolistsep]
		\item पन्हाळगडावरून पाळण्याचा बेतात शिवा न्हावी जाणीवपूर्वक मरणाला सामोरा गेला. बाजीप्रभू २००-३०० मावळ्यानबरोबर मरेपर्यंत लढला.
		\item मरण्याची प्रेरणा ध्येयवादातून येते. या ध्येयवादाला चारित्र्यवान नेत्याची जोड असेल तर त्याच्या शब्दाखातीर माणसे हसत मरू शकतात. हे माणसाच्या जातीचे वैशिष्ठय आहे. अशा वेळी तो नेता व्यक्ती राहिलेला नसतो तर तो जनमानसाचा प्रतिनिधी झालेला असतो. राष्ट्राच त्याचा तोंडून बोलत असते.
		\end{itemize}

	\item मिर्झा राज्यांनी १६६५ मध्ये ३ महिन्यात शिवाजीला पूर्ण खिळखिळे करून शरण आणले. या सरळ पराभवातून शिवाजी उभा राहतो ते सामर्थ्य तलवारीतून उगवणारे नसून मनातून उगवणारे आहे. “एका बलाढ्य शक्ती विरुद्ध तात्पुरती पड खाणे, पुढच्या उठावाच्या योजना आखणे, त्याच बरोबर जनतेची-सैन्याची उभारी कायम ठेवणे व त्यामुळे पुढचा उठाव नेत्रदीपक होईल याची काळजी घेणे”  हे त्याचे सूत्र होते.
	\end{itemize}

\item १६७० - : पुन:श्च उठाव
	\begin{itemize}[noitemsep,nolistsep]
	\item मोगली मुलुखात लुट. गेलेले किल्ले परत घेतले. यात तानाजी-निळोपंत-अण्णाजी दत्तो यांचे बलिदान. कोकणात उतरून मोगलांचे प्रभुत्व संपवले. सुरतेवर हल्ला.
	\item राज्याभिषेक. सर्व भारताला ‘आपण मुक्तिदाते आहोत’ असे आश्वासन होते.
	\item औरंगजेबाचे दक्षिणेत आगमन. महाराजांच्या मृत्युनंतर देखील जनतेने लढा चालूच ठेवला.
	\end{itemize}

\item आरमार
	\begin{itemize}[noitemsep,nolistsep]
	\item समुद्रामार्गील धोक्याकडे मुसलमानांनी लक्ष नाही दिले ते महाराजांनी दिले.
	\item युरोपच्या सागरी सामर्थ्यापुढे मराठ्यांचे आरमार किरकोळ होते तरी पराक्रम वाखाणण्यासारखा होता. सिद्दी, पोर्तुगीज आणि इंग्रज यांचा पाडाव करणे जमले नाही तरी नाकेबंदी मात्र निश्चित केली.
	\item पुढे नाना पेशव्यांच्या काळात मराठ्यांचे आरमार नष्ट करण्यात आले, त्यामुळे युरोपियननांना समूद्रावर एकमुखी सत्ता स्थापन करता आली व नंतर भारतावर ताबा मिळवणे शक्य झाले.
	\end{itemize}

\item नवीन शासन व्यवस्था: महाराजांबद्दल जनतेत अपरिमित आपुलकीचे कारण म्हणजे त्याने बसवलेली व्यवस्था.
	\begin{itemize}[noitemsep,nolistsep]
	\item वतने वंशपरंपरागत होती व त्याचा मोह प्रचंड होता. वतनदारीच्या रक्षणार्थ कत्तली व्हायच्या. पिढीजात वैरे. 
	\item पूर्वीचे वतनदार राजाला दुहेरी त्रासाचे असत. नवीन शत्रू आला तर त्याला सामील होऊन वतनाची हमी घेत. करवसुली बेसुमार करीत. कायद्याप्रमाणे नफ्याच्या १/६ कर असे पण वतनदार पाहिजे ते वसूल करून थोडाच वर राजाला पाठवित असे. पीक येवो न येवो जनता नागविली जात असे. आणि तक्रार कोणाला करणार? न्यायनिवडा करणारा पुन्हा वतनदारच होता. अब्रू सुरक्षित नव्हती. मुली पळविल्या जायच्या. प्रचंड फौजांच्या जाण्याचाही मोठा त्रास व्हायचा. प्रदेश उजाडला जायचा.
	\item शिवाजीच्या राज्याच्या यशाचे रहस्य हे वतनदारीच्या वधात आहे. त्यानी नवीन व्यवस्था आणली. जमिनीची पुन्हा मोजमापे केली. कोण कसणार ते ठरविले. उत्पन्नाचा अंदाज ठरविला. २/५ कर लाविला. वास्तविक हा पूर्वीपेक्षा जास्त होता पण जनतेने तो आनंदाने दिला कारण कर निश्चित झाले व काय नक्की पदरात पडणार आहे हे जनतेला कळाले. समाधानाने जगण्यापुरते उत्त्पन सुटू लागले. पक्षपात नाही. पूर्ण संरक्षण आणि निर्दोष अंमलबजावणी. याला जोड म्हणून मधूनमधून येणाऱ्या दुष्काळात त्याने करमाफी दिली. पीक येई पर्यंत अन्न दिले. बी-बियाणे नांगर दिले. ही सर्व व्यवस्था लावण्यात सर्वात मोठा व विश्वासू सहकारीअण्णाजी दत्तो (कुलकर्णी) होता. जनता सुखी झाली.
	\item शिवाजीने वतनदारांच्या फौजा काढून घेतल्या. वाडे तोडले. करावसुलीचे – न्यायनिवाड्याचे हक्क काढून घेतले. अपराध्यांचे हात-पाय तोडले. स्त्रीची अब्रू बिनधोक करून जनतेला पोटभर अन्न दिले. प्रजेची फौजांनी केलेली लुटमार दंडनीय ठरवली (एका प्रसंगात अश्या फौजेतील ३०० जणांचे हात पाय तोडले होते). शिवाजीने सिव्हील अॅथॉरिटीचा पुरस्कार केला. म्हणूनच रामदास त्याला ‘ शिवकल्याण राजा’ म्हणतात.
	\end{itemize}

\item	आदर्श: 
	\begin{itemize}[noitemsep,nolistsep]
	\item शिवाजी प्रजेचा कल्याणकर्ता, जनतेच्या स्वातंत्र्याचा पुरस्कर्ता होता. नैतिक बैठक निर्दोष होती. शत्रूची(ही) स्त्री, पूजास्थान, धर्मग्रंथ यांचा अवमान कधी केला नाही. सूड ही शिवाजीची प्रेरणा नव्हती.
	\item वैयक्तिक जीवनातील भोगलालसा त्याला नव्हती. त्याच्या प्रेरणा अध्यात्मिक होत्या. म्हणून रामदास त्याला ‘श्रीमंत योगी’ म्हणतात.
	\end{itemize}

\end{itemize}
