Chronology

\begin{itemize}[noitemsep,nolistsep]
\item Being written from 1632-80, 3 three updates
	\begin{itemize}[noitemsep,nolistsep]
	\item \textbf{जुना दासबोध /२१ समासी दासबोध}: १६३२. विवेक, वैराग्य, अध्यात्मिक अनुभव 
	\item  \textbf{७ दशकांचा }: विस्तारित आवृत्ती 
	\item \textbf{२० दशकांचा}: ७७५१ ओव्या
	\end{itemize}

\end{itemize}

Ideas
\begin{itemize}[noitemsep,nolistsep]
\item \textbf{Atma Tatva}: beyond the Creator (Brahma), Sustainer (Vishnu) and Destroyer (Mahesh/Shiva).
\item  \textbf{Worship path for the common man}: Sagun (with form, Atma) first, Nirgun (without form, Parmatma) next. To follow Nigun worship is not possible for everyone.
	\begin{itemize}[noitemsep,nolistsep]
	\item \textbf{Ramayan}: Laxman – Sagun bhakti, Bharat – Nirgun bhakti.
	\item  \textbf{Mahabharat}: Arjun – Sagun, Uddhav – Nirgun.
	\end{itemize}
\item After my death, fear not, worry not! I will always be alive amongst you via the teachings of DasBodh.	
\end{itemize}

Management (via questions)
\begin{itemize}[noitemsep,nolistsep]
\item \textbf{Why Samarth Ramdas is eligible to talk on management, leadership etc?} : He himself built a big organization of 1100 muttts, all over India, in such a way that the work  continues till today. The principles he followed, practices/techniques used are from his own workings (प्रचीती). Apart from spreading spirituality (Hinduism), developing men, supporting Shivaji's work, social up-liftment of women and downtrodden are some of the cornerstones of his work. आधी केले मग सांगितले 
\item  \textbf{Goals}: ``उत्कट, भव्य तेची घ्यावे, मिळीमिळीत अवघेची टाकावे| निस्पृहपणे विख्यात व्हावे| भूमंडळी||''  उच्च गुणवत्तेचे, work `above self'
\item \textbf{Leadership}: ``महंते महंत करावे| उक्ती बुद्धीने भरावे| जाणते करून विखरावे| नाना देसी ||'' Leaders should develop new (generation) leaders, motivate them, appoint them at various places, thats the expansion strategy.
\item \textbf{Qualities of Leader}: 
\end{itemize}
